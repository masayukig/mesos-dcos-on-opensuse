\documentclass[aspectratio=169,11pt,hyperref={colorlinks=true}]{beamer}
% https://github.com/zr-tex8r/BXcjkjatype/blob/master/README-ja.md
\usepackage[whole]{bxcjkjatype}
\usetheme{boxes}
\setbeamertemplate{navigation symbols}{}
\definecolor{suse}{RGB}{2, 211, 95}
\definecolor{susedark}{RGB}{13, 44, 64}
\setbeamercolor{titlelike}{fg=suse}
\setbeamercolor{structure}{fg=suse}
\hypersetup{colorlinks,urlcolor=suse}
\setbeamertemplate{footline}[frame number]
% Inserting graphics
\usepackage{graphicx}
% Side-by-side figures, etc
\usepackage{subfigure}
% Code snippits
\usepackage{listings}
% Color stuff
\usepackage{color}
% underline
\usepackage{soul}

\usepackage{amsmath}
\usepackage{tikz}
\newcommand\RBox[1]{%
  \tikz\node[draw,rounded corners,align=center,] {#1};%
}
\usepackage{hyperref}
%\usecolortheme{buzz}
%\usecolortheme{wolverine}
%\usetheme{Boadilla}
\usepackage[T1]{fontenc}
%\usepackage{fontspec}
%\setmainsfont{hiragino-elcapitan}
%\usepackage[expert, deluxe]{otf}

\definecolor{mygreen}{rgb}{0,0.6,0}
\definecolor{mygray}{rgb}{0.5,0.5,0.5}
\definecolor{mymauve}{rgb}{0.58,0,0.82}


%\usepackage{CJK}
%\pdfmapline{=genshingothic@Unicode@ <genshingothic.ttf}
% bxcjkjatype
%\setgothicfont[<ID>]{<フォントファイル名>}
%\setgothicfont{/Users/foo/Library/Fonts/genshingothic.ttf}
%\setgothicfont{/Users/foo/Library/Fonts/NotoSansCJKjp-Regular.otf}
%\setgothicfont{/Users/foo/Downloads/genshingothic-20150607/GenShinGothic-P-Normal.ttf}
%\setgothicfont{/Users/foo/Downloads/genshingothic-20150607/GenShinGothic-Regular.ttf}
%\setgothicfont{hiragino.ttc}
\setgothicfont{mplus-1p-regular.ttf}
\setCJKfamilydefault{\gtdefault}
%\setCJKfamilydefault{\mcdefault}
%\CJKforce{abcdefghijklmnopqrstuvwxyzABCDEFGHIJKLMNOPQRSTUVWXYZ}


\lstset{%
  backgroundcolor=\color{susedark},   % choose the background color; you must add \usepackage{color} or \usepackage{xcolor}
  breakatwhitespace=false,         % sets if automatic breaks should only happen at whitespace
  breaklines=true,                 % sets automatic line breaking
  captionpos=b,                    % sets the caption-position to bottom
  commentstyle=\color{suse},  % comment style
  extendedchars=true,              % lets you use non-ASCII characters; for 8-bits encodings only, does not work with UTF-8
  keepspaces=true,                 % keeps spaces in text, useful for keeping indentation of code (possibly needs columns=flexible)
  keywordstyle=\color{blue},       % keyword style
%  otherkeywords={*,...},           % if you want to add more keywords to the set
  numbersep=5pt,                   % how far the line-numbers are from the code
  numberstyle=\tiny\color{mygray}, % the style that is used for the line-numbers
  rulecolor=\color{white},         % if not set, the frame-color may be changed on line-breaks within not-black text (e.g. comments (green here))
  showspaces=false,                % show spaces everywhere adding particular underscores; it overrides 'showstringspaces'
  showstringspaces=false,          % underline spaces within strings only
  showtabs=false,                  % show tabs within strings adding particular underscores
  stringstyle=\color{suse},   % string literal style
}

\setbeamerfont{caption}{series=\normalfont,size=\fontsize{6}{8}}
%\setbeamerfont{caption}{series=\normalfont,size=\large}
\setbeamertemplate{caption}{\raggedright\insertcaption\par}

\setlength{\abovecaptionskip}{0pt}
\setlength{\floatsep}{0pt}

\author[Masayuki Igawa]{%
  \texorpdfstring{%
    \centering
    Masayuki Igawa\\
    \href{mailto:masayuki@igawa.io}{masayuki@igawa.io}\\
    \texttt{masayukig on Freenode,
     \href{https://twitter.com/masayukig}{Twitter},
     \href{https://github.com/masayukig}{GitHub}}
  }
  {Masayuki Igawa}
}
\date{May 30, 2017}

\title[Mesos, DC/OS on openSUSE
\hspace{2em}\insertframenumber/\inserttotalframenumber]{Mesos, DC/OS on openSUSE}

\setbeamercolor{background canvas}{bg=susedark}
\setbeamercolor{titlelike}{fg=white}
\setbeamercolor{structure}{fg=white}
\setbeamercolor{normal text}{fg=white}
\begin{document}
%\begin{CJK*}{UTF8}{hiragino-elcapitan}
%\begin{CJK*}{UTF8}{genshingothic}


{%
% \setbeamertemplate{background canvas}{\includegraphics[width=\paperwidth,height=\paperheight]{background_title.png}}
\setbeamertemplate{footline}{}
\setbeamercolor{background canvas}{bg=susedark}
\begin{frame}[noframenumbering]
  \hypersetup{colorlinks,urlcolor=suse}
  \setbeamercolor{author}{fg=white}
  \setbeamercolor{date}{fg=white}
  \titlepage{}
  \centering
  \href{https://github.com/masayukig/mesos-dcos-on-opensuse}{github.com/masayukig/mesos-dcos-on-opensuse}
\end{frame}
}

% \section{Agenda}
% \begin{frame}
%   \frametitle{Agenda}
%   \begin{itemize}
%     \item 自己紹介
%     \item やりたいこと
%     \item やってみた
%     \item どうなった?(困ったこと)
%     \item 解決方法
%     \item まとめ
%   \end{itemize}
% \end{frame}

\section{Introduction}
\begin{frame}
  \frametitle{自己紹介}
  \begin{itemize}
    \item 所属企業:HPE -> SUSE/ノベル株式会社
      \begin{itemize}
        \item QE(Quality Engineering)チーム所属
        \item チームメンバー日本人は私だけ。日本にいるのも私だけ!
        \item \href{http://www.geekwire.com/2016/struggling-keep-pace-cloud-hewlett-packard-enterprise-cuts-staff/}{GeekWire post}
        \item \href{https://www.hpe.com/us/en/newsroom/news-archive/statement/2016/11/HPE-Partners-with-SUSE-to-Provide-Best-in-Class-Hybrid-Cloud-Offerings.html}{HPE Partners with SUSE to Provide Best-in-Class Hybrid Cloud Offerings}
      \end{itemize}
    \item 業務活動内容: OpenStack QA 領域でアップストリームを通じた開発
      \begin{itemize}
        \item Tempest, OpenStack-Health, Subunit2SQL, Stackviz等のコアメンバ (≒ コミッタ?)
        \item \href{http://stackalytics.com/?user_id=igawa&release=all&metric=all}{stackalytics.com/?user\_id=igawa}
      \end{itemize}
  \end{itemize}
\end{frame}

\begin{frame}
  \frametitle{やりたいこと}
  \begin{itemize}
    \item DC/OSってなに?手元のマシンで使ってみたい!
    \item Mac -> openSUSE(Linux)
    \item Vagrant, OpenStack, etc.
  \end{itemize}
\end{frame}

\begin{frame}
  \frametitle{やってみた}
  \begin{itemize}
    \item openSUSE(Tumbleweed) インストール
    \item VirtualBox インストール
    \item Vagrant インストール
    \item DC/OS インストール
    \item 起動!
  \end{itemize}
\end{frame}

\begin{frame}
  \frametitle{どうなった?: openSUSE(Tumbleweed) インストール}
  \begin{itemize}
    \item openSUSE Tumbleweed って何? % https://www.opensuse.org/#Tumbleweed の絵を入れる
    \item USB メモリなんて持ってない % USBメモリの写真を入れる
    \item 容量足りない
    \item ブートしない % ブートしない写真を入れる
  \end{itemize}
\end{frame}

\begin{frame}
  \frametitle{どうなった?: VirtualBox/Vagrant インストール}
  それぞれの公式サイト(VirtualBox/Vagrant)に Tumbleweed 用のパッケージなし...
  \begin{itemize}
    \item VirtualBox: \lstinline[basicstyle=\ttfamily\footnotesize,columns=fixed]{zypper install virtualbox}
    \item Vagrant: \lstinline[basicstyle=\ttfamily\footnotesize,columns=fixed]{zypper install vagrant}
  \end{itemize}
\end{frame}

\begin{frame}
  \frametitle{どうなった?: DC/OS インストール}
  \begin{itemize}
    \item \href{https://dcos.io/docs/1.9/installing/local/}{Install DC/OS with Vagrant} に従い、\href{https://github.com/dcos/dcos-vagrant/}{master branch} を使用
    \begin{itemize}
      \item \lstinline[basicstyle=\ttfamily\footnotesize,columns=fixed]{cp VagrantConfig-1m-1a-1p.yaml VagrantConfig.yaml}
      \item \lstinline[basicstyle=\ttfamily\footnotesize,columns=fixed]{vagrant up}
    \end{itemize}
    \item すんなりインストールが進んで、4つのVMが起動した!
    \item 自動テスト(?)も通ったっぽい !
    \item \url{http://m1.dcos} にアクセス! → つながらない・・
  \end{itemize}
\end{frame}

\begin{frame}
  \frametitle{解決: DC/OS インストール}
  \begin{itemize}
    \item Host-only Networks issue
      \lstinputlisting[basicstyle=\ttfamily\footnotesize,frame=single,language=bash]{host-only-network-issue.txt}
  \end{itemize}
\end{frame}

\begin{frame}
  \frametitle{DC/OS on Vagrant on VirtualBox on openSUSE 起動\!}
    \begin{center}
      \includegraphics[width=.7\textwidth]{dcos_vms.png}
    \end{center}
\end{frame}

\subsection{More Information}
\begin{frame}
\frametitle{Where to get more information}
  \begin{itemize}
    \item \url{https://www.opensuse.org/\#Tumbleweed}
    \item \url{https://www.virtualbox.org/}
    \item \url{https://www.vagrantup.com/}
    \item \url{https://dcos.io/docs/1.9/installing/local/}
    \item \url{https://github.com/dcos/dcos-vagrant/}
  \end{itemize}
\end{frame}

%\end{CJK*}{UTF8}{hiragino-elcapitan}
%\end{CJK*}{UTF8}{genshingothic}
\end{document}
